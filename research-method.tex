\section{Research Method: Grounded Theory}
\label{ResearchMethod}

We followed Charmaz' approach to Grounded Theory \cite{Charmaz} which provides an iterative approach to data collection, data coding, and analysis resulting in an emergent theory. The two primary data sources were field notes collected during continuous participant observations of a 7.5 month project and interviews with Pivotal software engineers, designers, and product managers. Interviews were recorded, transcribed, coded, and analyzed using constant comparison. 

Grounded Theory immerses the researcher within the context of the research subject from the point of view of the participants. As the research progresses, Grounded Theory allows the researcher to \quotes{incrementally direct the data collection and theoretical ideas.} The theory provides a starting place for inquiry, not a specific goal known at the beginning of the research. As we interact with the data, the data influences how we progress and alters the research direction. When starting a grounded theory research study, the core question is \quotes{What is happening here?}' (Glaser, 1978) \cite{GlaserTheoreticalSensitivity}. Our initial core question was: \quotes{What is happening at Pivotal when it comes to software development?}

\subsection{Participants}
The primary researcher interviewed 21 engineers, product managers, and designers who had experience with Pivotal's software development process. Participants were not paid for their time. 
\subsection{Interviewing}
The primary researcher relied on \quotes{intensive interviews} which Charmaz summarizes as \quotes{open-ended yet directed, shaped yet emergent, and paced yet unrestricted} \cite{Charmaz}. The technique relies on open-ended questions. The purpose is for the researcher to enter into the participant's personal perspective within the context of the research question. The interviewer needs to abandon assumptions and their own personal presumptions in order to understand and explore the interviewee's perspective. Charmaz \cite{Charmaz} contrasts intensive interviews from informational interviews which endeavor to collect accurate `facts' and investigative interviews that attempt to reveal hidden intentions or expose practices and policies. 
 
The interviews were open-ended explorations starting with the question, ``please draw on this sheet of paper, your view of Pivotal's software development process." The interviewer specifically didn't force initial topics and merely followed the path of the interviewee. 

After the first round of interviews, analytical work revealed emergent categories which were then explored deeper in subsequent interviews. While exploring new emergent core categories, whenever possible, the researcher initiated subsequent interviews with a goal of not forcing the issue. For example, ``please draw your feelings about the code" often resulted in conversations about code ownership. 

After the interview, the interview was transcribed into a Word document with timecode stamps for each segment.

\subsection{Field Notes}
In addition to collecting data from interviews, the primary researcher collected field notes while working as a engineer. The field notes comprise of multiple paragraph entries recorded several times a week collected over a six month period. The field notes describe individual and collective actions, captures what participants defined as interesting or problematic, and include anecdotes and observations. 
\subsection{Initial Coding}
The primary researcher followed line-by-line coding as recommended by Charmaz \cite{Charmaz}. The line-by-line coding helps the researcher slow down and examine for nuanced interactions in the data. Based upon Charmaz's advice, the primary researcher adopted a coding scheme that was simple, direct, analytic, and spontaneous.  

After the initial coding, another researcher reviewed the initial codes while reading the transcripts and listening to the audio recording. During a weekly research collaboration meeting, any concerns were discussed and addressed. During these meetings, we recorded and transcribed into grounded theory memos any discussions about analysis or understanding the codes. Recording the sessions mitigated Glazer's concerns about missing possible memos or insights that are verbally discussed. \cite{GlaserTheoreticalSensitivity}

Once emergent themes arrive, then coding in later parts of the research were focused around the themes. The focused coding phase allowed the primary researcher to ``sort, synthesize, integrate, and organize large amounts of data."
\subsection{Constant Comparison, Focused Coding, and Memoing}
As data was collected and coded, the researcher placed codes into a spreadsheet organized based on focused codes. Only ideas shared by multiple interviewees earned their way into focused codes and subsequent analysis. The researcher compared new codes to existing codes for emergence of new categories. The primary researcher periodically audited each category by comparing the codes to each other and verifying the cohesion of the category. For complex categories, the codes were printed onto index cards, arranged and re-arranged until the emergence of logical categories.  The researcher captured the analysis of codes, examinations of theoretical plausibility, and insights in memos. 

\subsection{Theoretical Sampling}
As theoretical codes emerged, the researcher altered data collection for saturating core categories. When collective code ownership emerged as a core category, additional sampling was collected to identify which situations would increase or decrease a programmer's sense of ownership and which practices were required to enable collective code ownership.

\section{Research Context}
\label{ResearchContext}
\subsection{Organizational Context: Pivotal}
Pivotal provides solutions for cloud-based computing, big data, and agile development. Pivotal Cloud Foundry is an open source Platform as a Service either on-premise or hosted in the cloud. Pivotal Big Data Suite stores and analyzes multiple large data sets using Hadoop, Hawq, and Green Plum. Pivotal Labs provides agile developers and designers for startups and enterprise companies to transform their software development process. Pivotal Labs has 16 offices around the world.

Pivotal Lab's mission is to both deliver highly-crafted software products and provide a transformative experience for their client's engineering cultures. In order to change a developer's way of working, Pivotal combines the client's software engineers with Pivotal's engineers at a Pivotal office where they can experience Extreme Programming in an environment conducive for agile development. This experience is similar to the creation of the NUMMI plant where General Motors sent workers to Japan to learn Toyota's Production System \cite{Nummi}. For startups, Pivotal might be the first engineers working on the project. For enterprise clients, Pivotal provides additional engineering resources to accomplish new business goals. Sometimes, Pivotal helps transform engineering cultures when they no longer routinely deliver code.  

Common team sizes are six developers with a designer and a product manager. In the Palo Alto office, the number of developers on a project currently ranges from 2, 4, 6, 8, 10, 22, and 28. Larger projects are decomposed into smaller coordinating teams with one product manager per team and one or two designers per team. 

Commonly utilized technologies include Angular, Android, backbone, iOS, Java, Rails, React, and Spring often deployed onto Pivotal's Cloud Foundry. 

Pivotal Labs has followed Extreme Programming \cite{ExtremeProgramming2004} since the late 1990's. While each team is autonomous in making its own decisions as to what is best for a particular project, the company culture strongly suggests following all of the core practices of Extreme Programming including Pair Programming, Test Driven Development, Weekly Retrospectives, Daily Stand-ups, Prioritized Backlog, Whole Team ownership of the project and code base, plus Kanban's notion of work flowing through people.

% In addition to the lengthy intensive interviews, the primary researcher asked all of the software engineers, designers, and product managers at the Palo Alto office the question, \quotes{Given all of the values, principles, and practices of Pivotal, what do you think is the heart of the matter, what is core to all we do?} Table \ref{CorePractice} includes the various answers. The diversity of answers shows that there is no common understanding as to what is core. 

% \begin{table}[t]
% \renewcommand{\arraystretch}{1.3}
% \centering
% \caption{Asking \quotes{What is core to all we do?} shows that there is no clear understanding of fundamental}
% \label{CorePractice}
% \begin{tabular}{|p{3.10in}|}
% \hline
% empathy \\ \hline
% teamwork \\ \hline
% communication \\ \hline
% doing things the right way \\ \hline
% constant communication \\ \hline
% collaboration \\ \hline
% pairing, TDD \\ \hline
% TDD, agile planning, pair programming \\ \hline
% feedback, fast feedback loop \\ \hline
% kindness, no matter what you do, if you hurt people, that's not good. Software is built by humans. Act human \\ \hline
% user research and feedback \\ \hline
% delivery of value to the customer \\ \hline
% keeping clients paying us means that I have a job. Pairing has a very real impact in attracting clients. TDD has large impact on code quality. Once I leave pivotal, TDD is what I will take to my next job \\ \hline
% doing the right thing \\ \hline
% iteration practices drive our other practices. We do lean design. Build Measure Learn \\ \hline
% short feedback loops both at the project level and personal level when people giving me feedback \\ \hline
% empathy \\ \hline
% pairing, testing \\ \hline
% self reflection and  team retros \\ \hline
% doing the right thing \\ \hline
% enabling companies to build great software \\ \hline
% guaranteed repeatable success \\ \hline
% kindness, feedback loops, bias towards action \\
% \hline
% \end{tabular}
% \end{table}