\section{Theory Evaluation}
\label{TheoryEvaluation}

In assessing a Grounded Theory research study, Charmaz identifies four criteria for evaluating a grounded theory study: credibility, (\quotes{is there sufficient data}), originality, (\quotes{do the categories offer new insights}), resonance (\quotes{does the theory make sense to participants}), usefulness (\quotes{does the theory offer useful interpretations}) \cite{StolGTinSE}. 

\begin{itemize}
\item 
\textbf{credibility:}  The number of open ended interviews and the field notes from participant observation serve as a rich data set for the analysis. 
%The developer staffing for the project serves as a compelling illustration of the theory in practice

\item
\textbf{originality:} Extreme Programming discusses Pair Programming and Continuous Refactoring. This paper adds the overlapping pair rotation practice as a means of removing knowledge silos. This paper also broadens the definition of collective code ownership by acknowledging that a team may loose its sense of ownership when certain activities occur.

\item
\textbf{resonance:} When shown to participants, they understand why collective code ownership can be eroded and how to work against that erosion.

\item
\textbf{usefulness:} 
When shown to participants, they immediately understand the theory and better understand why Pivotal follows certain practices. 
\end{itemize}

\section{Threats to Validity}

\subsection{External Validity}

\textbf{Generalizability across situations:} a grounded theory limitation is that the theory emerges from a particular context and may not be applicable to other situations. This work analyzed software projects at the Silicon Valley office of Pivotal following Extreme Programming. The results might not be applicable to other teams in industry wanting collective code ownership or  following Extreme Programming. Replicating the results with other teams would mitigate this threat. 

\subsection{Internal Validity}
\textbf{Researcher bias:} a risk of the participant-observer technique is that the researcher may lose perspective and become biased by being a member of the team. An outside observer might see something the researcher missed. We mitigated this risk by recording interviews and with a colleague reviewing the coding process. 

\textbf{Prior knowledge bias:} with grounded theory prior knowledge can aid the researcher in looking at interesting research questions or create difficulties in blinding the researcher about possible explanations. \cite{GlaserIssues}. We mitigated this risk with a colleague reviewing the coding process. 